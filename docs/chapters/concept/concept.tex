\section{Configuration File}

\subsection{Definition}
All \ttt{files} inside \ttt{config/fuji} directory are named \ttt{configuration file}

\subsection{Types}
\begin{note}{The types of configuration files}
    \begin{description}
        \item[Control File] {A \ttt{control file} is used to control behaviours.}
        \begin{description}
            \item[Main-Control File] The \ttt{main-control file} refers to the \ttt{config/fuji/config.json} file, which is used to enable/disable a module.
            \item[Module-Control File] Some modules will have their own control file, which is used to control the behaviour of the module.
        \end{description}

        \item[User-Data File] User-data files are used to store the data generated by the user.
    \end{description}
\end{note}


\clearpage


\section{Module}

\subsection{Definition}
A \ttt{module} is used to provide a specific purpose.
\begin{example}{The purpose of modules}
    \begin{description}
        \item [ChatModule] provides chat-format customization.
        \item [TpaModule] provides \ttt{/tpa} command.
    \end{description}
\end{example}

\subsection{Properties}
The properties of a module are as follows:
\begin{description}
    \item [Can be disabled] You can disable any module completely in \ttt{main-control file} by setting the value of its \ttt{enable} key to \ttt{false}.
    \item [Can work standalone] The codes of a module is self-contained, there is no reference to other modules.
\end{description}

\subsection{Module Path}
A \ttt{module} is identified by a unique \ttt{module path}.
\begin{example}{What a module-path looks like?}
    The module path of the module \ttt{tpa} is \ttt{tpa}. \\
    The module path of the module \ttt{history} whose parent module is \ttt{chat}, is \ttt{chat.history}. \\
    You will see a list of \ttt{enabled modules} identified by their \ttt{module path} at the server-startup process.
\end{example}
A \ttt{module} can have \ttt{sub-module}.
The relationship between \ttt{parent-module} and \ttt{sub-module} is relative, and there is nothing special about \ttt{sub-module}.

\subsection{How to enable/disable a module}
You can enable/disable a module in \ttt{config/fuji/config.json} by setting the value of its \ttt{enable} key to \ttt{true}/\ttt{false}
\\
\\
A \ttt{module} will be enabled if the following conditions are met:
\begin{enumerate}
    \item The required dependency mods are installed.
    \item The value of its \ttt{enable} key is set to \ttt{true}.
    \item Its \ttt{parnet-module} is \ttt{enabled}.
\end{enumerate}

\begin{example}{How to enable a sub-module}
    To make the module \ttt{chat.display} enabled, you need to enable \ttt{chat} module first.
\end{example}


\section{Job}\label{sec:job}

\subsection{Definition}
A \ttt{job} is some things will be done \ttt{repeatedly}.

\subsection{Cron Expression as Trigger Rule}
A language named \ttt{cron language} is used to \ttt{define} when a \ttt{job} should be triggered.

\begin{tips}{Don't write cron expression by hand. Use generator!}
    A \ttt{cron expression} looks like \ttt{0 * * ? * *}, which means \ttt{trigger the job every minute}.

    You can use the generator to generate a \ttt{cron expression}:
    \url{https://www.freeformatter.com/cron-expression-generator-quartz.html}
\end{tips}