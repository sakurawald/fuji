\LevelOne{cleaner}

\LevelTwo{Purpose}
This module provides the entity cleaner to remove specified entities automatically.

\begin{note}{Only use this module to clean some edge-case entity}
    Since the vanilla minecraft also has a cleaner to remove the item stack in the ground, so it's recommended to only use this module to clean some weak-loading entities, like: the sand item stack \ldots
\end{note}


\LevelTwo{Command}
\LevelThree{/cleaner clean}
\begin{note}{The cleaner will keep silent if cleans nothing}
    If the cleaner cleans nothing, then it will keep silent.(Which means you will not see any message in console, or in-game chat)
\end{note}

\begin{tips}{See what is cleaned in cleaner broadcast.}
    Hover your mouse on the cleaner broadcast, you can see what has been removed.
\end{tips}

\LevelTwo{Configuration}
\begin{Configuration}
    \item[cron]{
        The cron used to define the job to trigger \ic{/cleaner clean}.
    }

    \item[key2age]{
        The key is translatable key, which you can query in \href{https://github.com/sakurawald/fuji-fabric/blob/dev/.github/files/en_us.json}{en\_us.json language file in minecraft 1.21.} \\
        The translatable key of entity starts with entity.minecraft. \\
        The translatable key of item starts with item.minecraft and block.minecraft. \\
        The age is the existence time of the entity, the unit of age is game tick, which means 20 age = 20 ticks = 1 second. \\
        The cleaner will only remove the entities whose translatable key equals key, and age greater equals the defined age, and the entity must not in the ignore list.

        \begin{example}{Clean the sand-block entity lives longer than 60sec}
            \begin{json}
                "block.minecraft.sand": 1200
            \end{json}
        \end{example}
    }

    \item[ignore]{
        Entities match the ignore list will not be cleaned.

        \begin{NestedList}
            \item[ignoreItemEntity]{
                Should we ignore all item entity.
            }

            \item[ignoreLivingEntity]{
                Should we ignore all living entity?\\
                If you want the cleaner to remove monster or animals, you should enable this option.
            }

            \item[ignoreNamedEntity]{
                Should we ignore named entity.(With name tag, or name changed by anvil.)
            }

            \item[ignoreEntityWithVehicle]{
                Like entity riding in some other entity, \eg minecraft, pig or spider
            }

            \item[ignoreEntityWithPassengers]
            \item[ignoreGlowingEntity]
            \item[ignoreLeashedEntity]
        \end{NestedList}
    }

\end{Configuration}

\begin{samepage}
    \begin{note}{The built-in safety rule}
        The cleaner will \tbf{always ignore} the following types:
        \begin{enumerate}
            \item player
            \item any block attached entity (\eg leash\_knot)
            \item any vehicle entity (\eg minecart, boat \ldots)
        \end{enumerate}
    \end{note}
\end{samepage}

