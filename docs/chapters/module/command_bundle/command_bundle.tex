\LevelOne{command\_bundle}\label{ch:command_bundle}

\LevelTwo{Purpose}
This module allows you to create bundle commands, input one command, output many commands.

\LevelTwo{Command}
\LevelThree{/command-bundle}
The root command of this module.


\LevelTwo{Feature}
\begin{enumerate}
    \item {a user-friendly DSL to create new custom commands easily, with the interoperation of \ttt{variables}, \ttt{placeholders} and \ttt{selectors}.}
    \item {support complex \ttt{command argument type}: \ttt{required argument}, \ttt{literal argument} and even the \ttt{optional argument} with a default value.
    }
    \item {a powerful type-system to ensure the \ttt{type-safe} input, with fully command suggestion.
        \begin{tips}{To query all type strings}
            \ic{/fuji inspect argument-type}
        \end{tips}
    }
    \item {register and un-register commands on the fly, without the server restart!
        \begin{example}{Reload the bundle commands}
            Each time you modify the configuration file, you should issue \ic{/fuji reload}, this will unregister all bundle commands in the server, and register the bundle commands defined in the file into the server.\\\\
            Also, you can use the \ic{/command-bundle un-register} and \ic{/command-bundle register} manually.
        \end{example}

    }
\end{enumerate}

\LevelTwo{Example}
\begin{example}{Combine commands into one command}
    \begin{json}
    {
        "requirement": {
        "level": 0,
        "string": null
    },
        "pattern": "composite-heal",
        "bundle": [
        "say before heal %player:name%",
        "run as fake-op %player:name% particle minecraft:heart ~ ~2 ~"
        "run as player %player:name% heal",
        "say after heal %player:name%"
        ]
    }
    \end{json}
\end{example}

\begin{example}{Transform a command}
    \begin{json}
    {
        "requirement": {
        "level": 4,
        "string": null
    },
        "pattern": "warn <player player-arg> <greedy greedy-arg>",
        "bundle": [
        "run as player %player:name% send-message $player-arg <red>You are warned: $greedy-arg"
        ]
    }
    \end{json}
\end{example}

\begin{tips}{Assign a string permission for a bundle command}
    See~\nameref{ch:permission}
\end{tips}


\LevelTwo{Reference}
\begin{enumerate}
    \item \url{https://www.gamergeeks.net/apps/minecraft/particle-command-generator}
    \item \url{https://learn.microsoft.com/en-us/minecraft/creator/documents/particleeffects?view=minecraft-bedrock-stable}
\end{enumerate}
