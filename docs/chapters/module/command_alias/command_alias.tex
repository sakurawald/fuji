\LevelOne{command\_alias}\label{ch:command_alias}

\LevelTwo{Purpose}
This module allows you to define command alias, which redirect to the existing command node.

\begin{note}{A command node is identified by path}
    See also: \url{https://minecraft.fandom.com/wiki/Commands}
\end{note}

\LevelTwo{Example}
\begin{example}{Shorten a existing command}
    The configuration create a command alias from \ic{/r} to \ic{/reply}

    \begin{json}
    {
        "from": [
        "r"
        ],
        "to": [
        "reply"
        ]
    }
    \end{json}

\end{example}

\begin{example}{Shorten a existing command}
    The configuration create a command alias from \ic{/sudo} to \ic{/run as fake-op}
    \begin{json}
    {
        "from": [
        "sudo"
        ],
        "to": [
        "run",
        "as",
        "fake-op"
        ]
    }
    \end{json}
\end{example}

\LevelTwo{What's more?}
\begin{tips}{How can I define complex commands?}
    As you can see, the command alias module only support to \ttt{redirect} a simple command into another simple command.
    In other words, you can only use this module to create \ttt{alias} to a \ttt{existing command node}.
    Also, it's not allow to define \ttt{variable} inside a \ttt{command alias}.\\\\
    If you want to define complex commands, use~\nameref{ch:command_bundle} module.
\end{tips}
