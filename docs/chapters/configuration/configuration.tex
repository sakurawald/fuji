\chapter{Configuration}\label{ch:configuration}
\section{Main-Control File}

\ss{common}
The \ttt{common} options inside \ttt{config/fuji/config.json} will influence \ttt{all modules}.

\sss{quartz}
Fuji use \ttt{quartz} library as scheduler, all the~\nameref{sec:job} are managed by quartz.
Quartz library use a language called \ttt{cron language} to define when to trigger a job.

\ssss{logger\_level}
The logger level for \ttt{quartz}.
The logger level from high to low are: OFF, FATAL, ERROR, WARN, INFO, DEBUG, TRACE, ALL.

\begin{example}{Enable all logs for quartz}
    Set the value to \ttt{ALL} to display all the messages from quartz.
    It's recommended to set at least \ttt{WARN} level, to avoid \ttt{console spam}.
\end{example}

\sss{backup}
Fuji will back up the \ttt{config/fuji} directory automatically before it loads any module.

\ssss{max\_slots}
How many \ttt{backup} should we keep?

\ssss{skip}
The list of \ttt{path resolver} to skip in backup.
\\
Insert \ttt{head} means skip the folder \ttt{config/fuji/head}.

\sss{language}
\ssss{default\_language}
The default language to use.
\\
Fuji also supports multi-language based on player's client-side language if the server is able to do so.
\\
You need to enable \ttt{language module} to let fuji respect client-side's language settings.
\\
Also, if the server can't support client-side's language, it will fallback to the \ttt{deafult\_language}



\clearpage
\section{Module-Control File}
You can read more about \ttt{module-control file} for each \ttt{module} in \nameref{ch:module}